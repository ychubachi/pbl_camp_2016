% Created 2016-09-15 木 14:27
\documentclass[uplatex,dvipdfmx,14pt,presentation,t]{beamer}
\input{style}
\usepackage[utf8]{inputenc}
\usepackage[T1]{fontenc}
\usepackage{fixltx2e}
\usepackage{graphicx}
\usepackage{longtable}
\usepackage{float}
\usepackage{wrapfig}
\usepackage{rotating}
\usepackage[normalem]{ulem}
\usepackage{amsmath}
\usepackage{textcomp}
\usepackage{marvosym}
\usepackage{wasysym}
\usepackage{amssymb}
\usepackage{hyperref}
\tolerance=1000
\usepackage{pxjahyper}
\usepackage{txfonts}
\usepackage{minted}
\usepackage{tikz}
\usetheme{default}
\author{産業技術大学院大学 \linebreak 中鉢 欣秀}
\date{2016-09-15}
\title{今年度PBL実施上の課題}
\hypersetup{
  pdfkeywords={},
  pdfsubject={},
  pdfcreator={Emacs 24.5.1 (Org mode 8.2.10)}}
\begin{document}

\maketitle
\begin{frame}{目次}
\tableofcontents
\end{frame}


\section{アクティブラーニング(AL)としてのPBL}
\label{sec-1}
\begin{frame}[label=sec-1-1]{\normalsize アクティブラーニングとは?(文部科学省)}
教員による一方向的な講義形式の教育とは異なり、学修者の能動的な学修への参加を取り入れ
た教授・学習法の総称。 \alert{学修者が能動的に学修} することによって、認知的、倫理的、社会的能力、
教養、知識、経験を含めた \alert{汎用的能力の育成} を図る。
 発見学習、問題解決学習、体験学習、調査学習等
 が含まれるが、
 \alert{教室内でのグループ・ディスカッション、ディベート、グループ・ワーク等}
も有効なアクティブ・ラーニングの方法である。
\end{frame}
\begin{frame}[label=sec-1-2]{「PBL」とアクティブラーニング}
\begin{block}{PBLとは}
\begin{itemize}
\item プレジェクトを通したチーム学習(TBL)
\item TBL: Team Based Learning
\end{itemize}
\pause
\end{block}
\begin{block}{PBLはアクティブラーニングと言える?}
\begin{itemize}
\item 言える.ただし条件がある
\end{itemize}
\pause
\end{block}
\begin{block}{その条件とは?}
\begin{itemize}
\item \alert{学習者が能動的に学修} しているならば
\item \alert{汎用的能力の育成} が行われているならば
\end{itemize}
\end{block}
\end{frame}

\section{ALとスクラム}
\label{sec-2}
\begin{frame}[label=sec-2-1]{「スクラム」とその難しさ}
\begin{block}{ソフトウェア開発手法であるスクラム}
\begin{itemize}
\item ルールは簡単
\item ガイドブックはわずか17ページ
\end{itemize}
\pause
\end{block}
\begin{block}{ガイドの記述は“宣言的”}
\begin{itemize}
\item チーム全体の効率と最適化
\item 開発チームは自己組織化される
\item 機能横断的である
\end{itemize}
\end{block}
\end{frame}

\begin{frame}[label=sec-2-2]{レトロスペクティブの重要性}
\begin{itemize}
\item スクラムでは一般的に,KPTと呼ばれる振り返り手法を実施
\begin{itemize}
\item やって良かったこと(Keep)
\item 問題となったこと(Problem)
\item やってみたいこと(Try)
\end{itemize}
\end{itemize}
\pause
\begin{itemize}
\item KPTを実施することが目的ではない
\begin{itemize}
\item チームを \alert{より良くする改善} が実施されるべき
\end{itemize}
\end{itemize}
\end{frame}

\section{ALとしてのPBLの評価について}
\label{sec-3}
\begin{frame}[label=sec-3-1]{プロダクトとプロジェクト}
\begin{block}{プロダクト vs プロジェクト}
\begin{itemize}
\item PBLの評価は \uline{プロダクト} だけではない
\item \alert{プロジェクト} の評価をするべき
\end{itemize}
\pause
\end{block}
\begin{block}{そもそもAIITの成績評価の軸は}
\begin{center}
\begin{tabular}{lll}
 & 活動 & 成果\\
\hline
質 &  & \\
量 &  & \\
\end{tabular}
\end{center}
\end{block}
\end{frame}
\begin{frame}[label=sec-3-2]{アクティブラーニングの評価は?}
\begin{itemize}
\item 現在研究中
\item ただし,「PBLの評価」より研究事例は多そう
\begin{itemize}
\item \href{http://eduview.jp/?p=1636&page=2}{アクティブ・ラーニングをどう評価すべきか〜西岡加名恵氏に聞く | eduview - Part 2}
\item \href{http://www.nucba.ac.jp/active-learning/entry-15216.html}{アクティブラーニングにおける評価方法とは? | 名古屋商科大学}
\end{itemize}
\end{itemize}
\end{frame}

\begin{frame}[label=sec-3-3]{今後に向けて}
\begin{itemize}
\item 今後は,PBLをチーム学習でのアクティブラーニングラーニングだと捉え直し,教授法の改善に役立てられないか研究したい
\end{itemize}
\end{frame}
% Emacs 24.5.1 (Org mode 8.2.10)
\end{document}